\documentclass[10pt,xcolor=pdflatex]{beamer}
\usepackage{newcent}
\usepackage[utf8]{inputenc}
%\usepackage[czech]{babel}
\usepackage{hyperref}
\usepackage{fancyvrb}
\usetheme{FIT}

%%%%%%%%%%%%%%%%%%%%%%%%%%%%%%%%%%%%%%%%%%%%%%%%%%%%%%%%%%%%%%%%%%
\title[FYO projekt]{Aberácie šošoviek}

\author[]{Roman Dobiáš}

\institute[]{Brno University of Technology, Faculty of Information Technology\\
Bo\v{z}et\v{e}chova 1/2. 612 66 Brno - Kr\'alovo Pole\\
xdobia11@stud.fit.vutbr.cz}

\date{May 15, 2019}
%\date{\today}
%\date{} % bez data

%%%%%%%%%%%%%%%%%%%%%%%%%%%%%%%%%%%%%%%%%%%%%%%%%%%%%%%%%%%%%%%%%%

\begin{document}

\frame[plain]{\titlepage}

\begin{frame}\frametitle{Aberácie šošoviek}
    \begin{itemize}
        \item čo to znamená ?
        \item ako vznikajú ?
        \item dôsledok ?
    \end{itemize}
\end{frame}

\begin{frame}\frametitle{Paraxialná optika}
    zjednodušené modely optických sústav pre malé uhly
    sin x $\sim$ x

    Snellov zákon - TODO

    Obrázok geometrickej šošovky
\end{frame}

\begin{frame}\frametitle{Realita}
    Upravený Schnellov zákon
    Upravený Schnellov zákon pre chromatické vlny



\bluepage{Thank You For Your Attention !}

\end{document}
